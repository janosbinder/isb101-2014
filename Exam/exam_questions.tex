% To compile this document
% graphics.off();rm(list=ls());library('knitr');knit('exam_questions.Rnw');  for(i in 1:2) system('R CMD pdflatex -shell-escape exam_questions.tex')

%setwd("~/Dropbox/Docs/Teaching/isb101/Exam")


\documentclass{article}\usepackage[]{graphicx}\usepackage[usenames,dvipsnames]{color}
%% maxwidth is the original width if it is less than linewidth
%% otherwise use linewidth (to make sure the graphics do not exceed the margin)
\makeatletter
\def\maxwidth{ %
  \ifdim\Gin@nat@width>\linewidth
    \linewidth
  \else
    \Gin@nat@width
  \fi
}
\makeatother

\definecolor{fgcolor}{rgb}{0.345, 0.345, 0.345}
\newcommand{\hlnum}[1]{\textcolor[rgb]{0.686,0.059,0.569}{#1}}%
\newcommand{\hlstr}[1]{\textcolor[rgb]{0.192,0.494,0.8}{#1}}%
\newcommand{\hlcom}[1]{\textcolor[rgb]{0.678,0.584,0.686}{\textit{#1}}}%
\newcommand{\hlopt}[1]{\textcolor[rgb]{0,0,0}{#1}}%
\newcommand{\hlstd}[1]{\textcolor[rgb]{0.345,0.345,0.345}{#1}}%
\newcommand{\hlkwa}[1]{\textcolor[rgb]{0.161,0.373,0.58}{\textbf{#1}}}%
\newcommand{\hlkwb}[1]{\textcolor[rgb]{0.69,0.353,0.396}{#1}}%
\newcommand{\hlkwc}[1]{\textcolor[rgb]{0.333,0.667,0.333}{#1}}%
\newcommand{\hlkwd}[1]{\textcolor[rgb]{0.737,0.353,0.396}{\textbf{#1}}}%

\usepackage{framed}
\makeatletter
\newenvironment{kframe}{%
 \def\at@end@of@kframe{}%
 \ifinner\ifhmode%
  \def\at@end@of@kframe{\end{minipage}}%
  \begin{minipage}{\columnwidth}%
 \fi\fi%
 \def\FrameCommand##1{\hskip\@totalleftmargin \hskip-\fboxsep
 \colorbox{shadecolor}{##1}\hskip-\fboxsep
     % There is no \\@totalrightmargin, so:
     \hskip-\linewidth \hskip-\@totalleftmargin \hskip\columnwidth}%
 \MakeFramed {\advance\hsize-\width
   \@totalleftmargin\z@ \linewidth\hsize
   \@setminipage}}%
 {\par\unskip\endMakeFramed%
 \at@end@of@kframe}
\makeatother

\definecolor{shadecolor}{rgb}{.97, .97, .97}
\definecolor{messagecolor}{rgb}{0, 0, 0}
\definecolor{warningcolor}{rgb}{1, 0, 1}
\definecolor{errorcolor}{rgb}{1, 0, 0}
\newenvironment{knitrout}{}{} % an empty environment to be redefined in TeX

\usepackage{alltt}


\RequirePackage{/Library/Frameworks/R.framework/Versions/3.1/Resources/library/BiocStyle/sty/Bioconductor}

\AtBeginDocument{\bibliographystyle{/Library/Frameworks/R.framework/Versions/3.1/Resources/library/BiocStyle/sty/unsrturl}}








\title{ISB101 Exam}
\usepackage{amsmath}
\usepackage{minted}
\usepackage{natbib}
\usepackage{mathpazo}
%\usepackage{enumerate}
\usepackage{soul}
\usepackage{cases}
\setlength{\parindent}{0cm}
\IfFileExists{upquote.sty}{\usepackage{upquote}}{} 

\begin{document}

\maketitle

\newcounter{ExerciseNr}
\addtocounter{ExerciseNr}{1}

\section{Exercises} \label{sec:exercises}

\subsection{Exercise \arabic{ExerciseNr}}

You need to copy a dataset (\verb+~/Documents/P53_expression_levels.tsv+) to a remote server called \verb+apple.uni.lux+ preferably to the directory of \verb+/home/projects1/P53_screening+. What command would you use?

\begin{description}
  \item[A] \verb+cp apple.uni.lux:/home/projects1/P53_screening/ ~/Documents/P53_expression_levels.tsv+
  \item[B] \verb+scp apple.uni.lux:/home/projects1/P53_screening/ ~/Documents/P53_expression_levels.tsv+
  \item[C] \verb+scp ~/Documents/P53_expression_levels.tsv apple.uni.lux:/home/projects1/P53_screening/+
  \item[D] \verb+scp apple.uni.lux:~/Documents/P53_expression_levels.tsv /home/projects1/P53_screening/+
\end{description}

To work remotely you need the necessary clients. These \verb+Putty+ and the \verb+WinSCP+ program. Our first example is to print \verb+Hello BioWorld!+. It should be done like:

\begin{minted}{sh}
echo "Hello BioWorld!"
\end{minted}

\end{document}
