% To compile this document
% graphics.off();rm(list=ls());library('knitr');knit('EDA-lab.Rnw');  for(i in 1:2) system('R CMD pdflatex EDA-lab.tex')

%detach(bodyfat);

% extract R-code
% purl('EDA-lab.Rnw')

%setwd("/Volumes/Macintosh Storage/Users/jbinder/Dropbox/Docs/Teaching/isb101/Visualization in R/Tutorial")


\documentclass{article}\usepackage[]{graphicx}\usepackage[usenames,dvipsnames]{color}
%% maxwidth is the original width if it is less than linewidth
%% otherwise use linewidth (to make sure the graphics do not exceed the margin)
\makeatletter
\def\maxwidth{ %
  \ifdim\Gin@nat@width>\linewidth
    \linewidth
  \else
    \Gin@nat@width
  \fi
}
\makeatother

\definecolor{fgcolor}{rgb}{0.345, 0.345, 0.345}
\newcommand{\hlnum}[1]{\textcolor[rgb]{0.686,0.059,0.569}{#1}}%
\newcommand{\hlstr}[1]{\textcolor[rgb]{0.192,0.494,0.8}{#1}}%
\newcommand{\hlcom}[1]{\textcolor[rgb]{0.678,0.584,0.686}{\textit{#1}}}%
\newcommand{\hlopt}[1]{\textcolor[rgb]{0,0,0}{#1}}%
\newcommand{\hlstd}[1]{\textcolor[rgb]{0.345,0.345,0.345}{#1}}%
\newcommand{\hlkwa}[1]{\textcolor[rgb]{0.161,0.373,0.58}{\textbf{#1}}}%
\newcommand{\hlkwb}[1]{\textcolor[rgb]{0.69,0.353,0.396}{#1}}%
\newcommand{\hlkwc}[1]{\textcolor[rgb]{0.333,0.667,0.333}{#1}}%
\newcommand{\hlkwd}[1]{\textcolor[rgb]{0.737,0.353,0.396}{\textbf{#1}}}%

\usepackage{framed}
\makeatletter
\newenvironment{kframe}{%
 \def\at@end@of@kframe{}%
 \ifinner\ifhmode%
  \def\at@end@of@kframe{\end{minipage}}%
  \begin{minipage}{\columnwidth}%
 \fi\fi%
 \def\FrameCommand##1{\hskip\@totalleftmargin \hskip-\fboxsep
 \colorbox{shadecolor}{##1}\hskip-\fboxsep
     % There is no \\@totalrightmargin, so:
     \hskip-\linewidth \hskip-\@totalleftmargin \hskip\columnwidth}%
 \MakeFramed {\advance\hsize-\width
   \@totalleftmargin\z@ \linewidth\hsize
   \@setminipage}}%
 {\par\unskip\endMakeFramed%
 \at@end@of@kframe}
\makeatother

\definecolor{shadecolor}{rgb}{.97, .97, .97}
\definecolor{messagecolor}{rgb}{0, 0, 0}
\definecolor{warningcolor}{rgb}{1, 0, 1}
\definecolor{errorcolor}{rgb}{1, 0, 0}
\newenvironment{knitrout}{}{} % an empty environment to be redefined in TeX

\usepackage{alltt}


\RequirePackage{/Library/Frameworks/R.framework/Versions/3.1/Resources/library/BiocStyle/sty/Bioconductor}

\AtBeginDocument{\bibliographystyle{/Library/Frameworks/R.framework/Versions/3.1/Resources/library/BiocStyle/sty/unsrturl}}








\title{EDA--lab}
\usepackage{amsmath}
\usepackage{natbib}
\usepackage{mathpazo}
%\usepackage{enumerate}
\usepackage{soul}
\usepackage{cases}
\setlength{\parindent}{0cm} 

\author{Janos Binder$^1$ \\[1em]European Molecular Biology Laboratory (EMBL),\\ Heidelberg, Germany\\
\texttt{$^1$janos.binder@embl.de}}
\IfFileExists{upquote.sty}{\usepackage{upquote}}{}
\begin{document}

\maketitle


\tableofcontents

\section{Acknowledgements}

This tutorial is based on the material of Bernd Klaus (\href{http://www-huber.embl.de/users/klaus/main.html}{http://www-huber.embl.de/users/klaus/main.html)}.

\section{Required Packages and other Preparations} \label{sec:prep}

%
\begin{knitrout}
\definecolor{shadecolor}{rgb}{0.969, 0.969, 0.969}\color{fgcolor}\begin{kframe}
\begin{alltt}
\hlkwd{load}\hlstd{(}\hlkwd{url}\hlstd{(}\hlstr{"http://www-huber.embl.de/users/klaus/BasicR/seqZyx.rda"}\hlstd{))}
\hlkwd{library}\hlstd{(}\hlstr{"TeachingDemos"}\hlstd{)}
\hlkwd{data}\hlstd{(golub,} \hlkwc{package} \hlstd{=} \hlstr{"multtest"}\hlstd{)}
\hlkwd{library}\hlstd{(biomaRt)}
\hlkwd{library}\hlstd{(reshape2)}
\hlkwd{library}\hlstd{(ggplot2)}
\hlkwd{library}\hlstd{(plyr)}
\hlkwd{library}\hlstd{(xlsx)}
\hlkwd{library}\hlstd{(vioplot)}
\end{alltt}
\end{kframe}
\end{knitrout}
%

Should you get error messages, you can also install them quickly:

%
\begin{knitrout}
\definecolor{shadecolor}{rgb}{0.969, 0.969, 0.969}\color{fgcolor}\begin{kframe}
\begin{alltt}
\hlkwd{source}\hlstd{(}\hlstr{"http://bioconductor.org/biocLite.R"}\hlstd{)}
\hlstd{packs} \hlkwb{<-} \hlkwd{c}\hlstd{(}\hlstr{"TeachingDemos"}\hlstd{,} \hlstr{"multtest"}\hlstd{,} \hlstr{"biomaRt"}\hlstd{,}
           \hlstr{"reshape2"}\hlstd{,} \hlstr{"ggplot2"}\hlstd{,} \hlstr{"plyr"}\hlstd{,} \hlstr{"xlsx"}\hlstd{,} \hlstr{"vioplot"}\hlstd{)}
\hlkwd{biocLite}\hlstd{(packs)}
\end{alltt}
\end{kframe}
\end{knitrout}
%


\section{Introduction}

In this lab, a few essential methods are given to display and visualize data. 
It quickly answers questions like: How are my data distributed? How can the frequencies 
of nucleotides from a gene be visualized? Are there outliers in my data?
Does the distribution of my data resemble that of a bell--shaped curve? Are
there differences between gene expression values taken from two groups of
patients?\\

The most important central tendencies (mean, median) are defined and
illustrated together with the most important measures of spread (standard
deviation, variance, inter quartile range, and median absolute deviation).

We also introduce \CRANpkg{ggplot2} a package to produce elegant graphics
for data analysis.

\section{Base graphics and ggplot2}

The package \Rpackage{ggplot2} is one  of the most commonly used graphics packages for \R.
It is an alternative to the basic graphic system of \R, which is limited in various
aspects. In this lab we will use basic graphics in \R \vspace{2pt} and corresponding
 \Rpackage{ggplot2} plots side by side.

\Rpackage{ggplot2} is meant to be an implementation of the Grammar of Graphics, 
developed by L. Wilkinson, hence gg--plot. 
The basic notion is that there is a grammar to the composition of graphical components in 
statistical graphics, and by directly controlling that grammar, you can generate 
a large set of carefully constructed graphics tailored to your particular needs. 


The central concept of the approach is that 
plots convey information through various aspects of their aesthetics.
Aesthetics are mappings from the data to something you can visually perceive.
Some aesthetics that plots use are:

\begin{itemize}
  \item x position
  \item y position
  \item size of elements
  \item shape of elements
  \item color of elements
\end{itemize}

The elements in a plot are geometric shapes, like

\begin{itemize}
   \item points
   \item lines
   \item line segments
   \item bars
   \item text
\end{itemize}

     

Some of these geometries have their own particular aesthetics. For instance:

\begin{itemize}
    \item points
\begin{itemize}
     \item point shape
     \item point size
\end{itemize}
   \item lines
\begin{itemize}
      \item  line type
      \item  line weight
\end{itemize}
   \item bars
\begin{itemize}
      \item  y minimum
      \item  y maximum
      \item  fill color
      \item  outline color
\end{itemize}
    \item text
\begin{itemize}
       \item label value
\end{itemize}

\end{itemize}

Each component is added to the plot as a layer, hence you might start with 
a simple mapping of the raw data to the x-- and y--axes, creating a scatterplot.
A second layer may the be added by coloring the points according to a
group they belong to and so on.

There are other basics of these graphics that you can adjust, 
like the scaling of the aesthetics, and the positions of the geometries.

The values represented in the plot are the product of various statistics. If you just plot the 
raw data, you can think of each point representing the identity statistic. 
Many bar charts represent the mean or the median statistic. Histograms are bar charts 
where the bars represent the binned count or density statistics and so on.

\subsection{Building a Plot Layer by Layer}

There's a quick plotting function in \Rpackage{ggplot2} called \Rfunction{qplot()}
which is meant to be similar to the \Rfunction{plot()} function from base graphics. 
You can do a lot with \Rfunction{qplot()}, but it can be better to approach
the package from the layering syntax.

All \Rpackage{ggplot2} plots begin with the function \Rfunction{ggplot()}. 
\Rfunction{ggplot()} takes two primary 
arguments, \Robject{data} is the data frame containing the data to be plotted
and \Rfunction{aes( )} are the aesthetic mappings to pass on to the plot elements.

As you can see, the second argument, \Rfunction{aes()}, isn't a normal argument, but another 
function. Since we'll never use \Rfunction{aes()} as a separate function, it might be best
to think of it as a special way to pass a list of arguments to the plot.

The first step in creating a plot is to add one or more layers. Let's start with 
the iris data set as an example. Note that \Rpackage{ggplot2} always requires the specification
of the data frame from which the variables used in the plot are drawn.


\begin{knitrout}
\definecolor{shadecolor}{rgb}{0.969, 0.969, 0.969}\color{fgcolor}\begin{kframe}
\begin{alltt}
    \hlkwd{summary}\hlstd{(iris)}
\end{alltt}
\begin{verbatim}
#>     Sepal.Length   Sepal.Width    Petal.Length   Petal.Width        Species  
#>    Min.   :4.30   Min.   :2.00   Min.   :1.00   Min.   :0.1   setosa    :50  
#>    1st Qu.:5.10   1st Qu.:2.80   1st Qu.:1.60   1st Qu.:0.3   versicolor:50  
#>    Median :5.80   Median :3.00   Median :4.35   Median :1.3   virginica :50  
#>    Mean   :5.84   Mean   :3.06   Mean   :3.76   Mean   :1.2                  
#>    3rd Qu.:6.40   3rd Qu.:3.30   3rd Qu.:5.10   3rd Qu.:1.8                  
#>    Max.   :7.90   Max.   :4.40   Max.   :6.90   Max.   :2.5
\end{verbatim}
\begin{alltt}
    \hlstd{p} \hlkwb{<-} \hlkwd{ggplot}\hlstd{(iris,} \hlkwd{aes}\hlstd{(Sepal.Length, Sepal.Width) )}
\end{alltt}
\end{kframe}
\end{knitrout}

If you just type \Rcode{p} or \Rcode{print(p)}, you'll get back a warning saying
that the plot lacks  any layers. With the \Rfunction{ggplot()} function, 
we've set up a plot which is going to draw from the iris data, 
the Sepal.length variable will be mapped to the x--axis, and the Sepal.width
variable is going to be mapped to the y--axis. 
However, we have not determined which kind of geometric object will represent the data. 
Let's add points, for a scatterplot.

\begin{knitrout}
\definecolor{shadecolor}{rgb}{0.969, 0.969, 0.969}\color{fgcolor}\begin{kframe}
\begin{alltt}
 \hlstd{p} \hlopt{+} \hlkwd{geom_point}\hlstd{()}
\end{alltt}
\end{kframe}
\includegraphics[width=\maxwidth]{figure/ggplot-ex2} 

\end{knitrout}

Alternatively, this plot could have been produced with \Rfunction{qplot}.
Additionally, you can map color to the species. 


\begin{knitrout}
\definecolor{shadecolor}{rgb}{0.969, 0.969, 0.969}\color{fgcolor}\begin{kframe}
\begin{alltt}
\hlkwd{qplot}\hlstd{(Sepal.Length, Sepal.Width,} \hlkwc{data} \hlstd{= iris,} \hlkwc{color} \hlstd{= Species)}
\end{alltt}
\end{kframe}
\includegraphics[width=\maxwidth]{figure/ggplot-ex-qplot} 

\end{knitrout}

We clearly see that the  setosa plants have different Sepal.Length/Sepal.Width 
relationship compared to the other two species. The full documentation for \Rpackage{ggplot2} can be found at \url{http://docs.ggplot2.org/current/}.
Apart from mapping data to aesthetics, \Rpackage{ggplot2} 
can handle statistical transformations of the data, i.e. easily create all 
the nice exploratory graphics we will look at below.

We will explore one of these transformations by adding a regression line
to the data of each of the three plant species as a third layer.


\begin{knitrout}
\definecolor{shadecolor}{rgb}{0.969, 0.969, 0.969}\color{fgcolor}\begin{kframe}
\begin{alltt}
\hlstd{ggsmooth} \hlkwb{<-} \hlstd{(}\hlkwd{qplot}\hlstd{(Sepal.Length, Sepal.Width,} \hlkwc{data} \hlstd{= iris,} \hlkwc{color} \hlstd{= Species)}
\hlopt{+} \hlkwd{stat_smooth}\hlstd{(}\hlkwc{method} \hlstd{=} \hlstr{"lm"}\hlstd{))}
\hlstd{ggsmooth}
\end{alltt}
\end{kframe}
\includegraphics[width=\maxwidth]{figure/ggplot-ex-smoother} 

\end{knitrout}

The command \Rfunction{stat\textunderscore smooth} first adds a statistical transformation
to the existing data and then plots it using a certain geometry, in this
case a special "smooth" geometry that is tailored to the plotting of regression
fits. You can obtained the statistical transformations by looking at the saved
plot and extracting the appropriate sublist. 

\begin{knitrout}
\definecolor{shadecolor}{rgb}{0.969, 0.969, 0.969}\color{fgcolor}\begin{kframe}
\begin{alltt}
\hlstd{transformed.data} \hlkwb{<-} \hlkwd{as.list}\hlstd{(}\hlkwd{print}\hlstd{(ggsmooth))}\hlopt{$}\hlstd{data[[}\hlnum{2}\hlstd{]]}
\end{alltt}
\end{kframe}
\end{knitrout}

Thus, you could also map the transformed data differently than the default geometry
does it. This however does not make much sense in this case.

\subsection{Some biological data}

Here we're looking at some real biological data: different doses of HFG (a cytokine)
were applied to cells and the downstream effect to the phosphorylation of  target
proteins were assessed recording a time course--signal in different conditions.
This data and the exercise ideas were provided by Lars Velten (Steinmetz lab). 
We first  load  the dataset.

\begin{knitrout}
\definecolor{shadecolor}{rgb}{0.969, 0.969, 0.969}\color{fgcolor}\begin{kframe}
\begin{alltt}
\hlstd{proteins}\hlkwb{<-}\hlkwd{read.csv}\hlstd{(}\hlstr{"http://www-huber.embl.de/users/klaus/BasicR/proteins.csv"}\hlstd{)[,}\hlopt{-}\hlnum{1}\hlstd{]}
\hlkwd{head}\hlstd{(proteins)}
\end{alltt}
\begin{verbatim}
#>       Condition min Target    Signal    Sigma
#>   1 10ng/mL HGF   0   pAKT -6.71e+08 96749276
#>   2 10ng/mL HGF   5   pAKT  5.68e+08 97144057
#>   3 10ng/mL HGF  10   pAKT  1.05e+09 97215659
#>   4 10ng/mL HGF  20   pAKT -3.23e+08 97055787
#>   5 10ng/mL HGF  30   pAKT -9.00e+08 96908309
#>   6 10ng/mL HGF  60   pAKT  3.17e+08 96731346
\end{verbatim}
\begin{alltt}
\hlstd{proteins_pMek} \hlkwb{<-} \hlkwd{subset}\hlstd{(proteins, proteins}\hlopt{$}\hlstd{Target} \hlopt{==} \hlstr{"pMEK"}\hlstd{)}
\hlstd{proteins_pMek_sub} \hlkwb{<-} \hlkwd{subset}\hlstd{(proteins_pMek, proteins_pMek}\hlopt{$}\hlstd{Condition} \hlopt{==} \hlstr{"10ng/mL HGF"}\hlstd{)}
\end{alltt}
\end{kframe}
\end{knitrout}

We can start simple by only looking at the first condition 
of the "pMEK" protein target for now. We simply
produce a line plot of the signal across time. In this plot
we use the data  \Robject{proteins\textunderscore pMek\textunderscore sub},
 map \Robject{min}
to the x--axis and \Robject{Signal} to the y--axis and use a 
line a a geometry. Additional modifications are added to the plot
in the exercise.

\begin{knitrout}
\definecolor{shadecolor}{rgb}{0.969, 0.969, 0.969}\color{fgcolor}\begin{kframe}
\begin{alltt}
\hlstd{proteins_pMek_sub}
\end{alltt}
\begin{verbatim}
#>         Condition min Target    Signal    Sigma
#>   103 10ng/mL HGF   0   pMEK  3.38e+08 31005696
#>   104 10ng/mL HGF   5   pMEK -2.09e+08 31400418
#>   105 10ng/mL HGF  10   pMEK  7.20e+07 31199120
#>   106 10ng/mL HGF  20   pMEK -2.76e+08 31015194
#>   107 10ng/mL HGF  30   pMEK  8.70e+08 31140886
#>   108 10ng/mL HGF  60   pMEK  2.42e+08 31040783
#>   109 10ng/mL HGF 120   pMEK  4.31e+07 31072343
\end{verbatim}
\begin{alltt}
\hlkwd{qplot}\hlstd{(min, Signal,} \hlkwc{data} \hlstd{= proteins_pMek_sub,} \hlkwc{geom} \hlstd{=} \hlstr{"line"}\hlstd{)}
\end{alltt}
\end{kframe}
\includegraphics[width=\maxwidth]{figure/plot-protein-data} 
\begin{kframe}\begin{alltt}
\hlcom{# or}
\hlcom{#ggplot(proteins_pMek_sub, aes(min, Signal)) + geom_line()}
\end{alltt}
\end{kframe}
\end{knitrout}

\subsubsection*{Exercise: Simple ggplot usage}
 
Using the data \Robject{proteins\textunderscore pMek\textunderscore sub}, 
do the following

\begin{enumerate}[label=(\emph{\alph*})]

\item Use points as a geometry instead of lines
\item Use both lines and points
\item Add errorbars \Robject{geom\textunderscore errorbar} to the plot.
This requires  further aesthetics: 
\Robject{ymax} and \Robject{ymin}. 
The estimated error is stored in the variable \Robject{Sigma}.

\end{enumerate}



We can also easily plot all conditions for the protein pMEK by
mapping color to the experimental condition of pMEK.


\begin{knitrout}
\definecolor{shadecolor}{rgb}{0.969, 0.969, 0.969}\color{fgcolor}\begin{kframe}
\begin{alltt}
\hlkwd{qplot}\hlstd{(min, Signal,} \hlkwc{data} \hlstd{= proteins_pMek,} \hlkwc{geom} \hlstd{=} \hlstr{"line"}\hlstd{,} \hlkwc{color} \hlstd{= Condition)}
\end{alltt}
\end{kframe}
\includegraphics[width=\maxwidth]{figure/plot-protein-data-2} 
\begin{kframe}\begin{alltt}
\hlcom{# or}
\hlcom{#ggplot(proteins_pMek, aes(min, Signal, color = Condition) ) + geom_line()}
\end{alltt}
\end{kframe}
\end{knitrout}

\subsection{Setting axis limits}
Another important aspect of data display are the limits of the 
x-- and y--axis. You can change these limits by using \Robject{xlim()}
and \Robject{ylim()}. However, the overall scale will not be "retrained"
and only be limited. This essentially results in plotting only a subset
of the data, excluding values outside of the 
limits. In order to actually "zoom--in" without excluding data points
one has to use the  \Rfunction{coord\textunderscore cartesian()} command. In the
code below, we first limit the y--axis to 1e9, excluding all
other data points, then we retrain the scale.

\begin{knitrout}
\definecolor{shadecolor}{rgb}{0.969, 0.969, 0.969}\color{fgcolor}\begin{kframe}
\begin{alltt}
\hlcom{## limit scale}
\hlstd{(}\hlkwd{qplot}\hlstd{(min, Signal,} \hlkwc{data} \hlstd{= proteins_pMek,} \hlkwc{geom} \hlstd{=} \hlstr{"line"}\hlstd{,} \hlkwc{color} \hlstd{= Condition)}
\hlopt{+} \hlkwd{ylim}\hlstd{(}\hlkwd{c}\hlstd{(}\hlnum{0}\hlstd{,}\hlnum{1e9}\hlstd{)))}
\end{alltt}


{\ttfamily\noindent\color{warningcolor}{\#>\ \  Warning: Removed 4 rows containing missing values (geom\_path).}}\end{kframe}
\includegraphics[width=\maxwidth]{figure/limitng_vs_zooming1} 
\begin{kframe}\begin{alltt}
\hlcom{## retrain scale}
\hlstd{(}\hlkwd{qplot}\hlstd{(min, Signal,} \hlkwc{data} \hlstd{= proteins_pMek,} \hlkwc{geom} \hlstd{=} \hlstr{"line"}\hlstd{,} \hlkwc{color} \hlstd{= Condition)}
\hlopt{+} \hlkwd{coord_cartesian}\hlstd{(}\hlkwc{ylim} \hlstd{=} \hlkwd{c}\hlstd{(}\hlnum{0}\hlstd{,} \hlnum{1e9}\hlstd{)))}
\end{alltt}
\end{kframe}
\includegraphics[width=\maxwidth]{figure/limitng_vs_zooming2} 

\end{knitrout}

\subsection{Faceting}

\Rpackage{ggplot2} also allows you to easily split plots according to a factor 
variable, plotting parts of the data in different panels. Returning
to the iris data, we can easily plot different species in different 
panels using \Rfunction{facet\textunderscore wrap()}. This is
the simplest faceting function, splitting according to one
factor only. 


\begin{knitrout}
\definecolor{shadecolor}{rgb}{0.969, 0.969, 0.969}\color{fgcolor}\begin{kframe}
\begin{alltt}
\hlstd{ggsmooth} \hlopt{+} \hlkwd{facet_wrap}\hlstd{(} \hlopt{~} \hlstd{Species)}
\end{alltt}
\end{kframe}
\includegraphics[width=\maxwidth]{figure/iris-panels} 

\end{knitrout}

The splitting is defined in a formula notation:
yfactor \textasciitilde xfactor. Factors can also be combined,
e.g.  yfactor \textasciitilde xfactor\textunderscore 1 + xfactor\textunderscore 2. 
Faceting  can also be used with \Rfunction{qplot()}, there you always have
to specify a two--sided formula. If you only want to use a single
splitting factor, you can use the dot notation:

\begin{knitrout}
\definecolor{shadecolor}{rgb}{0.969, 0.969, 0.969}\color{fgcolor}\begin{kframe}
\begin{alltt}
\hlstd{(}\hlkwd{qplot}\hlstd{(Sepal.Length, Sepal.Width,} \hlkwc{data} \hlstd{= iris,}
\hlkwc{color} \hlstd{= Species,} \hlkwc{facets} \hlstd{= .} \hlopt{~}  \hlstd{Species))}
\end{alltt}
\end{kframe}
\includegraphics[width=\maxwidth]{figure/iris-panels_qplot} 

\end{knitrout}


\subsubsection*{Exercise: ggplot faceting}
 
Using the data \Robject{proteins\textunderscore pMek}, 
produce a plot split by the experimental condition factor using 
\Rfunction{facet\textunderscore wrap()}. 



\subsection{Melting and casting data frames}  

The data table we loaded was already suitable for the plots we wanted 
to produce since every line represents exactly one observation. 
However, this is not necessarily the case and we might want to represent
the different time points by different columns, not just a single one.

In time series analysis parlance our current data would be in "long" format,
but we might want to transform it into a "wide" format, with a separate 
column for every time point. The package \CRANpkg{reshape2} allows 
you to do this. \Rpackage{ggplot2} usually requires "long" formats, which can be
obtained by using the function \Rfunction{melt}. Thus a "molten"
data frame in \Rpackage{ggplot2} corresponds to a "long" format.  "wide" formats
can be computed using the function \Rfunction{dcast}. As an example
we will now represent every time point of our data frame as a single
column. 

Note that the wide format is only compatible with a single numerical target
variable, so we only include signal as a variable here. The
casting grid is defined by a formula with the variables that will
be represented by a single column on the left (x--variables) and 
the wide variables on the right side of the tilde (y--variables). 

\begin{knitrout}
\definecolor{shadecolor}{rgb}{0.969, 0.969, 0.969}\color{fgcolor}\begin{kframe}
\begin{alltt}
\hlstd{proteins_cast} \hlkwb{<-} \hlkwd{dcast}\hlstd{(proteins, Target} \hlopt{+} \hlstd{Condition} \hlopt{~}   \hlstd{min ,}
                \hlkwc{value.var} \hlstd{=} \hlstr{"Signal"}\hlstd{)}
\hlkwd{head}\hlstd{(proteins_cast)}
\end{alltt}
\begin{verbatim}
#>     Target          Condition         0        5        10        20        30      45        60
#>   1   pAKT        10ng/mL HGF -6.71e+08 5.68e+08  1.05e+09 -3.23e+08 -9.00e+08      NA  3.17e+08
#>   2   pAKT        40ng/mL HGF -7.59e+08 3.32e+08  1.95e+08  1.29e+08  7.43e+07      NA  1.01e+09
#>   3   pAKT 40ng/mL HGF + AKTi -2.03e+08 1.62e+08 -2.41e+08        NA -4.69e+08      NA -3.31e+08
#>   4   pAKT 40ng/mL HGF + MEKi  7.30e+08 6.57e+08  8.73e+08        NA  1.39e+09      NA  7.10e+08
#>   5   pAKT        80ng/mL HGF -3.09e+08 8.36e+08  8.91e+08  1.12e+09  9.08e+08 3.8e+08 -6.95e+08
#>   6  pERK1        10ng/mL HGF  2.77e+08 2.37e+08 -3.36e+08 -1.48e+08  7.06e+08      NA  2.85e+07
#>           120
#>   1 -4.69e+08
#>   2 -7.95e+08
#>   3  5.92e+08
#>   4 -2.21e+08
#>   5 -2.16e+08
#>   6  5.52e+08
\end{verbatim}
\end{kframe}
\end{knitrout}

We can now melt the data frame again. For melting, you have to specify
id variable and/or measurement variables. id variables identify a single 
line in the molten data frame and measured variables represent the 
measurements. If you only supply one of them, it is assumed that
all the other variables belong to the other group. Columns
that are not in either of the groups are discarded.

A factor column is then added to indicate to which
former column a measurement belongs to. In our case this is the
time point. Another useful function is \Rfunction{arrange}
which allows you to reorder the data frame according
to certain columns.  

\begin{knitrout}
\definecolor{shadecolor}{rgb}{0.969, 0.969, 0.969}\color{fgcolor}\begin{kframe}
\begin{alltt}
\hlstd{proteins_molten} \hlkwb{<-} \hlkwd{melt}\hlstd{(proteins_cast,} \hlkwc{id.vars} \hlstd{=} \hlkwd{c}\hlstd{(}\hlstr{"Target"}\hlstd{,} \hlstr{"Condition"}\hlstd{),}
\hlkwc{variable.name}\hlstd{=}\hlstr{"Time"}\hlstd{,} \hlkwc{value.name}\hlstd{=} \hlstr{"Signal"}\hlstd{)}
\hlstd{proteins_molten} \hlkwb{<-}\hlkwd{arrange}\hlstd{(proteins_molten, Condition)}
\hlkwd{head}\hlstd{(proteins_molten)}
\end{alltt}
\begin{verbatim}
#>     Target   Condition Time    Signal
#>   1   pAKT 10ng/mL HGF    0 -6.71e+08
#>   2  pERK1 10ng/mL HGF    0  2.77e+08
#>   3  pERK2 10ng/mL HGF    0 -1.91e+08
#>   4   pMEK 10ng/mL HGF    0  3.38e+08
#>   5   pAKT 10ng/mL HGF    5  5.68e+08
#>   6  pERK1 10ng/mL HGF    5  2.37e+08
\end{verbatim}
\end{kframe}
\end{knitrout}

\subsubsection*{Exercise: complex ggplot example}
 
Use the data frame \Robject{proteins} to produce 
a plot of the time courses  split by the experimental 
target and colored according to the experimental 
conditions. Add error bars to your plot.





%ldply(list(value = proteins[,1:4], value = proteins[,c(1:3,5)] ))

\section{Univariate Data Display}

In order to study the distribution of data, various visualization methods are 
available. We will look at some of them in the following.

\subsection{Frequency Table and Barplot}

Discrete data occur when the values naturally fall into categories. A frequency 
table simply gives the number of occurrences within a category.

\subsubsection*{Example: Nucleotide Frequencies } 
A gene consists of a sequence of nucleotides $\{A, C, G, T \}$.
The number of each nucleotide can be displayed in a frequency table. This
will be illustrated by an Exon of the Zyxin gene, which plays an important role in cell
adhesion. You can use the Bioconductor package \Rpackage{biomaRt} to query
various genomic databases. The commented code below was originally used to retrieve the
exon sequence. The command \Rfunction{strplit} splits the sequence into
its single letter parts. We already loaded the sequence at the beginning of
the lab.


%
\begin{knitrout}
\definecolor{shadecolor}{rgb}{0.969, 0.969, 0.969}\color{fgcolor}\begin{kframe}
\begin{alltt}
\hlstd{ensembl}\hlkwb{=}  \hlkwd{useMart}\hlstd{(}\hlstr{"ensembl"}\hlstd{,}\hlkwc{dataset}\hlstd{=}\hlstr{"hsapiens_gene_ensembl"}\hlstd{)}
\hlstd{seqZyx} \hlkwb{<-} \hlkwd{getSequence}\hlstd{(}\hlkwc{id} \hlstd{=} \hlstr{"ENSG00000159840"}\hlstd{,} \hlkwc{type} \hlstd{=} \hlstr{"ensembl_gene_id"}\hlstd{,}
\hlkwc{mart} \hlstd{= ensembl,} \hlkwc{seqType} \hlstd{=} \hlstr{"gene_exon"}\hlstd{)[}\hlnum{1}\hlstd{,]}
\hlstd{seqZyx} \hlkwb{<-} \hlkwd{strsplit}\hlstd{(}\hlkwd{as.character}\hlstd{(seqZyx[}\hlnum{1}\hlstd{,}\hlnum{1}\hlstd{]),} \hlkwc{split} \hlstd{=} \hlkwd{character}\hlstd{(}\hlnum{0}\hlstd{))}
\hlstd{seqZyx} \hlkwb{<-} \hlstd{seqZyx[[}\hlnum{1}\hlstd{]]}
\hlkwd{save}\hlstd{(seqZyx,} \hlkwc{file} \hlstd{=} \hlstr{"seqZyx.rda"}\hlstd{)}
\end{alltt}
\end{kframe}
\end{knitrout}
%

A (frequency) table, a corresponding pie plot  and a barplot
can be produced by following commands

%
\begin{knitrout}
\definecolor{shadecolor}{rgb}{0.969, 0.969, 0.969}\color{fgcolor}\begin{kframe}
\begin{alltt}
\hlkwd{table}\hlstd{(seqZyx)} \hlcom{## table}
\end{alltt}
\begin{verbatim}
#>   seqZyx
#>    A  C  G  T 
#>   29 94 70 30
\end{verbatim}
\begin{alltt}
\hlkwd{prop.table}\hlstd{(}\hlkwd{table}\hlstd{(seqZyx))} \hlcom{## frequency table}
\end{alltt}
\begin{verbatim}
#>   seqZyx
#>       A     C     G     T 
#>   0.130 0.422 0.314 0.135
\end{verbatim}
\begin{alltt}
\hlkwd{barplot}\hlstd{(}\hlkwd{table}\hlstd{(seqZyx))}
\end{alltt}
\end{kframe}
\includegraphics[width=\maxwidth]{figure/Zyx-table} 

\end{knitrout}

In \Rpackage{ggplot2} we first have to transform our data into an
appropriate \Robject{data.frame}. We then map the bases to 
the x--axis and add binning by using the bar geometry.

\begin{knitrout}
\definecolor{shadecolor}{rgb}{0.969, 0.969, 0.969}\color{fgcolor}\begin{kframe}
\begin{alltt}
\hlstd{ggBarplot} \hlkwb{=} \hlkwd{ggplot}\hlstd{(} \hlkwc{data} \hlstd{=} \hlkwd{data.frame}\hlstd{(}\hlkwc{bases} \hlstd{=} \hlkwd{factor}\hlstd{(seqZyx)),} \hlkwd{aes}\hlstd{(}\hlkwc{x} \hlstd{= bases))}
\hlstd{ggBarplot} \hlkwb{=} \hlstd{ggBarplot} \hlopt{+} \hlkwd{xlab}\hlstd{(}\hlstr{""}\hlstd{)} \hlopt{+} \hlkwd{geom_bar}\hlstd{()}
\hlstd{ggBarplot}
\end{alltt}
\end{kframe}
\includegraphics[width=\maxwidth]{figure/Zyx-table-ggplot2} 

\end{knitrout}
%

In order to plot a pie chart with \Rpackage{ggplot2}, we first have 
to create a stacked bar plot by  mapping the base counts to the fill 
of the bar and then use polar coordinates.

\begin{knitrout}
\definecolor{shadecolor}{rgb}{0.969, 0.969, 0.969}\color{fgcolor}\begin{kframe}
\begin{alltt}
\hlkwd{pie}\hlstd{(}\hlkwd{table}\hlstd{(seqZyx))}
\end{alltt}
\end{kframe}
\includegraphics[width=\maxwidth]{figure/Zyx-pie1} 
\begin{kframe}\begin{alltt}
\hlstd{ggPie} \hlkwb{=} \hlkwd{ggplot}\hlstd{(} \hlkwc{data} \hlstd{=} \hlkwd{data.frame}\hlstd{(}\hlkwc{bases} \hlstd{=} \hlkwd{factor}\hlstd{(seqZyx)),} \hlkwd{aes}\hlstd{(}\hlkwc{x} \hlstd{=} \hlkwd{factor}\hlstd{(}\hlnum{1}\hlstd{),}  \hlkwc{fill} \hlstd{= bases))}
\hlstd{ggPie} \hlkwb{=} \hlstd{ggPie} \hlopt{+} \hlkwd{xlab}\hlstd{(}\hlkwa{NULL}\hlstd{)} \hlopt{+} \hlkwd{geom_bar}\hlstd{()} \hlopt{+} \hlkwd{scale_x_discrete}\hlstd{(}\hlkwc{breaks}\hlstd{=}\hlkwa{NULL}\hlstd{)}
\hlstd{ggPie} \hlopt{+} \hlkwd{coord_polar}\hlstd{(}\hlkwc{theta} \hlstd{=} \hlstr{"y"}\hlstd{)}
\end{alltt}
\end{kframe}
\includegraphics[width=\maxwidth]{figure/Zyx-pie2} 

\end{knitrout}
%

\subsection{Scatterplots and Stripcharts}
An elementary method to visualize data is the so--called scatterplot,
which simply plots two sets of data against each other or, if applied to a single
data set, plots each data point of the set according to its index. \\

A stripchart is similar to a scatterplot, but plots all data points of a single
data set horizontally or vertically. This is particularly useful in combination with a factor 
that distinguishes members from different experimental conditions or patients groups.


\subsubsection*{Example: Visualizing Gene CCND3} 
Many visualization methods will be illustrated by the Golub
 microarray data on two sub types of leukemia. We shall concentrate on the expression values of gene
"CCND3 Cyclin D3", which are collected in row 1042 of the data matrix
\Rfunction{golub}. To plot the data values one can simply use \Rfunction{plot(golub[1042,])}. In
the resulting plot  the vertical axis gives the size of the expression
values and the horizontal axis the index of the patients. This is a scatterplot 
of a single data set. \\

 It can be observed
that the values for patient 28 to 38 are somewhat lower, but, indeed, the
picture is not very clear because the groups are not plotted separately.
To produce two adjacent stripcharts one for the ALL and one for the
AML patients, we use a factor called  \Rfunction{gol.fac} which
separates the classes of patients.

\begin{knitrout}
\definecolor{shadecolor}{rgb}{0.969, 0.969, 0.969}\color{fgcolor}\begin{kframe}
\begin{alltt}
\hlkwd{plot}\hlstd{(golub[}\hlnum{1042}\hlstd{,])}
\end{alltt}
\end{kframe}
\includegraphics[width=\maxwidth]{figure/plotCCND3} 

\end{knitrout}

\begin{knitrout}
\definecolor{shadecolor}{rgb}{0.969, 0.969, 0.969}\color{fgcolor}\begin{kframe}
\begin{alltt}
\hlstd{gol.fac} \hlkwb{<-} \hlkwd{factor}\hlstd{(golub.cl,}\hlkwc{levels}\hlstd{=}\hlnum{0}\hlopt{:}\hlnum{1}\hlstd{,} \hlkwc{labels}\hlstd{=} \hlkwd{c}\hlstd{(}\hlstr{"ALL"}\hlstd{,}\hlstr{"AML"}\hlstd{))}
\hlkwd{stripchart}\hlstd{(golub[}\hlnum{1042}\hlstd{,]} \hlopt{~} \hlstd{gol.fac,} \hlkwc{method}\hlstd{=}\hlstr{"jitter"}\hlstd{,} \hlkwc{vertical} \hlstd{=} \hlnum{TRUE}\hlstd{)}
\end{alltt}
\end{kframe}
\includegraphics[width=\maxwidth]{figure/stripchartCCND3} 

\end{knitrout}

From the resulting figure it can be observed that the CCND3
expression values of the ALL patients tend to have larger expression values
than those of the AML patients. The option \Robject{jitter} will ``smear''
the data points a little bit. In \Rpackage{ggplot2} these plots can be created
like this:

\begin{knitrout}
\definecolor{shadecolor}{rgb}{0.969, 0.969, 0.969}\color{fgcolor}\begin{kframe}
\begin{alltt}
\hlstd{CCND3} \hlkwb{<-} \hlkwd{data.frame}\hlstd{(}\hlkwc{expCCND3} \hlstd{= golub[}\hlnum{1042}\hlstd{,],} \hlkwc{patients} \hlstd{= gol.fac)}

\hlstd{ggScatter} \hlkwb{=} \hlkwd{qplot}\hlstd{(}\hlkwd{seq_along}\hlstd{(expCCND3), expCCND3,} \hlkwc{data} \hlstd{= CCND3,} \hlkwc{geom} \hlstd{=}\hlstr{"point"}\hlstd{)}
\hlstd{ggScatter} \hlopt{+} \hlkwd{xlab}\hlstd{(}\hlstr{"Index"}\hlstd{)}
\end{alltt}
\end{kframe}
\includegraphics[width=\maxwidth]{figure/ggplot2scatter1} 
\begin{kframe}\begin{alltt}
\hlstd{ggStrip} \hlkwb{=}  \hlkwd{ggplot}\hlstd{(CCND3,} \hlkwd{aes}\hlstd{(}\hlkwc{x}\hlstd{=patients,} \hlkwc{y}\hlstd{=expCCND3))}
\hlstd{ggStrip} \hlopt{+} \hlkwd{geom_point}\hlstd{(}\hlkwc{position} \hlstd{=}  \hlkwd{position_jitter}\hlstd{(}\hlkwc{w} \hlstd{=} \hlnum{.0}\hlstd{,} \hlkwc{h} \hlstd{=} \hlnum{.30}\hlstd{))}
\end{alltt}
\end{kframe}
\includegraphics[width=\maxwidth]{figure/ggplot2scatter2} 

\end{knitrout}

The ``jittering'' in \Rpackage{ggplot2} occurs after the plot has been created
and the data has been mapped to aesthetics that is why it is called a ``position
adjustment''.  Note that the degree of jittering can be controlled 
explicitly in \Rpackage{ggplot2}.


\subsection{Histograms}

Another method to visualize data is by dividing the range of data values into
a number of intervals and to plot the frequency per interval as a bar. Such
a plot is called a histogram.

\subsubsection*{Example: Histogram of CCND3}
A histogram of the expression values of gene CCND3 
of the acute lymphoblastic leukemia patients can be produced as follows:

\begin{knitrout}
\definecolor{shadecolor}{rgb}{0.969, 0.969, 0.969}\color{fgcolor}\begin{kframe}
\begin{alltt}
\hlkwd{hist}\hlstd{(golub[}\hlnum{1042}\hlstd{, gol.fac}\hlopt{==}\hlstr{"ALL"}\hlstd{])}
\end{alltt}
\end{kframe}
\includegraphics[width=\maxwidth]{figure/histCCND3} 

\end{knitrout}

The function \Rfunction{hist} divides the data into 5 intervals having width equal to
0.5. Observe  that one value is small and the
other ones are more or less symmetrically distributed around the mean. The number
of bins can be adjusted by the \Rfunction{breaks} option, choosing
\Rfunction{prob = TRUE}, will give the relative frequency for each bin on
the $y$--axis instead of the absolute frequencies. \\


\Rpackage{ggplot2} uses binning as the default geometry if only x--values are
supplied, so creating a histogram is really easy. By default it uses the range of
the data divided by 30 as the binwidth. This can be easily adjusted. Here, we
produce a barplot given the absolute counts per bin and then a classical histogram
giving the relative frequencies.

In the code below, \Rfunction{(aes(y = ..density..)} means
that we map the y--coordinate to the relative frequency 
estimate as returned by the \Rfunction{stat \textunderscore bin} 
statistical transformation used  by the ``geom\textunderscore histogram'' 
function.

\begin{knitrout}
\definecolor{shadecolor}{rgb}{0.969, 0.969, 0.969}\color{fgcolor}\begin{kframe}
\begin{alltt}
\hlstd{CCND3.ALL} \hlkwb{=} \hlkwd{data.frame}\hlstd{(}\hlkwc{CCND3.ALL} \hlstd{= golub[}\hlnum{1042}\hlstd{, gol.fac}\hlopt{==}\hlstr{"ALL"}\hlstd{])}
\hlstd{ggplotAbsFreq} \hlkwb{=} \hlkwd{qplot}\hlstd{(golub[}\hlnum{1042}\hlstd{, gol.fac}\hlopt{==}\hlstr{"ALL"}\hlstd{],} \hlkwc{binwidth} \hlstd{=} \hlnum{0.5}\hlstd{)}
\hlstd{ggplotHist} \hlkwb{=} \hlkwd{ggplot}\hlstd{(CCND3.ALL,} \hlkwd{aes}\hlstd{(}\hlkwc{x}\hlstd{=CCND3.ALL ))}
\hlstd{ggplotHist} \hlkwb{=} \hlstd{ggplotHist} \hlopt{+} \hlkwd{geom_histogram}\hlstd{(}\hlkwd{aes}\hlstd{(}\hlkwc{y} \hlstd{= ..density..),} \hlkwc{binwidth} \hlstd{=} \hlnum{0.5}\hlstd{)}


\hlstd{ggplotAbsFreq}
\end{alltt}
\end{kframe}
\includegraphics[width=\maxwidth]{figure/ggplot2hist1} 
\begin{kframe}\begin{alltt}
\hlstd{ggplotHist}
\end{alltt}
\end{kframe}
\includegraphics[width=\maxwidth]{figure/ggplot2hist2} 

\end{knitrout}


\subsection{Kernel density estimates}
Kernel density estimates are a method to turn the histogram
into a smooth density estimate. Given data
$x_1, \dotsc, x_n$ and a constant called "bandwidth" $h$,
the kernel density estimate is given by:

\begin{center}
\[
\hat f(x)= \frac{1}{n}\sum_{i=1}^{n}  \frac{1}{h}K\left( \frac{x-x_i}{h} \right) 
\]
\end{center}

Typical kernels are:
\begin{itemize}
\item Bisquare kernel: $K(u) = \frac{15}{16}(1-u^2)^2 \; \text{ for } u \in [-1,1] \; \text{and} \; 0 \text{ otherwise }$
\item Gau{\ss} kernel: $K(u) = \frac{1}{\sqrt{2\pi}} \exp\Biggl(-\frac{1}{2} u^2  \Biggr) \; \text{ for } \; u \in \mathbb{R}$ 
\end{itemize}

They readily computed using the \R  \hspace{2pt} function \Rfunction{density} or the appropriate
ggplot statistical transformation, which uses this function.

\begin{knitrout}
\definecolor{shadecolor}{rgb}{0.969, 0.969, 0.969}\color{fgcolor}\begin{kframe}
\begin{alltt}
\hlstd{CCND3.ALL} \hlkwb{=} \hlkwd{data.frame}\hlstd{(}\hlkwc{CCND3.ALL} \hlstd{= golub[}\hlnum{1042}\hlstd{, gol.fac}\hlopt{==}\hlstr{"ALL"}\hlstd{])}
 \hlkwd{qplot}\hlstd{(golub[}\hlnum{1042}\hlstd{, gol.fac}\hlopt{==}\hlstr{"ALL"}\hlstd{],} \hlkwc{geom} \hlstd{=} \hlstr{"density"}\hlstd{)}
\end{alltt}
\end{kframe}
\includegraphics[width=\maxwidth]{figure/ggplot2density} 

\end{knitrout}

Since we do not have many data points, the kernel density estimates 
gives a somewhat different impression than the histogram. Also note 
that kernel density estimates do not respect the boundaries of the
data set by default.

\subsection{Boxplots}
It is always possible to sort $n$ data values to have increasing order $x_1 \leq 
x_2 \leq \dots \leq x_n$ , where $x_1$ is the smallest, $x_2$ is the first--to--the 
smallest, etc. 

Let $x_{0.25}$ be a number for which it holds that 25\% of the data values 
$x_1 , \dotsc, x_n$ are smaller. That is, 25\% of the data values 
lay on the left side of the number $x_{0.25}$, the reason for which it is called the first quartile or the 25th percentile. \\


The second quartile is the value $x_{0.5}$ such that 50\% of the data values are
smaller. It is also called the median. 
Similarly, the third quartile or 75th percentile is the value $x_{0.75}$ such
that 75\% of the data is smaller.\\

A popular method to display data is by drawing a box around the first and the 
third quartile, a bold line segment for the median, and the
smaller line segments (whiskers) for the smallest and
the largest data values. Such a data display is known as a box--and--whisker
plot or simply boxplot.


\subsubsection*{Example: Boxplot of CCND3}
A view on the distribution of the expression values of the
ALL and the AML patients on gene CCND3  can be obtained by
constructing two separate boxplots adjacent to one another. To produce such
a plot the factor gol.fac is again very useful.
%
\begin{knitrout}
\definecolor{shadecolor}{rgb}{0.969, 0.969, 0.969}\color{fgcolor}\begin{kframe}
\begin{alltt}
\hlkwd{boxplot}\hlstd{(golub[}\hlnum{1042}\hlstd{,]} \hlopt{~} \hlstd{gol.fac)}
\end{alltt}
\end{kframe}
\includegraphics[width=\maxwidth]{figure/boxplotCCND3} 

\end{knitrout}
%

In \Rpackage{ggplot2} a boxplot can be exactly as a stripchart,
we just have to change the geometry of the plot:

\begin{knitrout}
\definecolor{shadecolor}{rgb}{0.969, 0.969, 0.969}\color{fgcolor}\begin{kframe}
\begin{alltt}
\hlstd{ggBox}\hlkwb{=}  \hlkwd{ggplot}\hlstd{(CCND3,} \hlkwd{aes}\hlstd{(}\hlkwc{x}\hlstd{=patients,} \hlkwc{y}\hlstd{=expCCND3,} \hlkwc{color} \hlstd{= patients))}
\hlstd{ggBox} \hlopt{+} \hlkwd{geom_boxplot}\hlstd{()}
\end{alltt}
\end{kframe}
\includegraphics[width=\maxwidth]{figure/ggBox} 

\end{knitrout}

From the position of the boxes,  it can be observed that the gene
expression values for ALL are larger than those for AML. Furthermore, since
the two sub-boxes around the median are more or less equally wide, the data
are quite symmetrically distributed around the median. The quantiles can
be computed with the function  \Rfunction{quantile}. By default the four
``quartiles'' are returned but other quantiles can be obtained by specifying
the \Rfunction{probs} parameter

\begin{knitrout}
\definecolor{shadecolor}{rgb}{0.969, 0.969, 0.969}\color{fgcolor}\begin{kframe}
\begin{alltt}
\hlkwd{quantile}\hlstd{(golub[}\hlnum{1042}\hlstd{,])}
\end{alltt}
\begin{verbatim}
#>       0%    25%    50%    75%   100% 
#>   -0.743  1.043  1.813  2.049  2.766
\end{verbatim}
\begin{alltt}
\hlcom{## custom quantiles}
\hlkwd{quantile}\hlstd{(golub[}\hlnum{1042}\hlstd{,],} \hlkwc{probs} \hlstd{=} \hlkwd{c}\hlstd{(}\hlnum{0}\hlstd{,} \hlnum{0.1}\hlstd{,} \hlnum{0.5}\hlstd{))}
\end{alltt}
\begin{verbatim}
#>       0%    10%    50% 
#>   -0.743  0.484  1.813
\end{verbatim}
\end{kframe}
\end{knitrout}

Outliers are data values laying far apart from the pattern set by the
majority of the data values. The implementation in \Rfunction{R} of the (modified)
boxplot draws such outlier points separately as small circles. A data point
$x$ is defined as an outlier point if
\begin{gather*}
x < x_{0.25} - 1.5 \cdot (x_{0.75} - x_{0.25} )
\intertext{or} 
x > x_{0.75} + 1.5  \cdot (x_{0.75} - x_{0.25} )
\end{gather*}

From the boxplot above it can be observed that there are outliers among the gene
expression values of ALL patients. These are the smaller values 0.45827 and
1.10546, and the largest value 2.76610. The AML expression values have one
outlier with value -0.74333.
To define extreme outliers, the factor 1.5 is raised to 3.0.  
These numbers can be conveniently extracted from the 
\Rfunction{boxplot.stats} function:

\begin{knitrout}
\definecolor{shadecolor}{rgb}{0.969, 0.969, 0.969}\color{fgcolor}\begin{kframe}
\begin{alltt}
\hlkwd{boxplot.stats}\hlstd{(golub[}\hlnum{1042}\hlstd{, gol.fac} \hlopt{==} \hlstr{"ALL"}\hlstd{])}
\end{alltt}
\begin{verbatim}
#>   $stats
#>   [1] 1.28 1.80 1.93 2.18 2.59
#>   
#>   $n
#>   [1] 27
#>   
#>   $conf
#>   [1] 1.81 2.04
#>   
#>   $out
#>   [1] 2.766 1.105 0.458
\end{verbatim}
\begin{alltt}
\hlcom{### outliers are in the "out" element of the list }
\end{alltt}
\end{kframe}
\end{knitrout}

\subsection{Violin plots}

A violin plot is a combination of a boxplot and a kernel density estimate. 
Specifically, instead of straight borders for the boxes, a kernel density estimate
is displayed. 

\begin{knitrout}
\definecolor{shadecolor}{rgb}{0.969, 0.969, 0.969}\color{fgcolor}\begin{kframe}
\begin{alltt}
\hlkwd{vioplot}\hlstd{(}\hlkwd{split}\hlstd{(golub[}\hlnum{1042}\hlstd{,],  gol.fac)[[}\hlnum{1}\hlstd{]],} \hlkwd{split}\hlstd{(golub[}\hlnum{1042}\hlstd{,],  gol.fac)[[}\hlnum{2}\hlstd{]],}
        \hlkwc{names} \hlstd{=} \hlkwd{c}\hlstd{(}\hlstr{"ALL"}\hlstd{,} \hlstr{"AML"}\hlstd{))}
\end{alltt}
\end{kframe}
\includegraphics[width=\maxwidth]{figure/Violin_plot_example1} 
\begin{kframe}\begin{alltt}
\hlstd{ggBox} \hlopt{+} \hlkwd{geom_violin}\hlstd{()}
\end{alltt}
\end{kframe}
\includegraphics[width=\maxwidth]{figure/Violin_plot_example2} 

\end{knitrout}


\subsection{Empirical Cumulative Distribution Function}
Another way to visualize data is the empirical cumulative distribution function 
(ecdf). It is an estimator for the actual  cumulative distribution function,
which gives the probability of observing values less than a given value $t$.
Especially if you have a large number
of data points, it gives a useful visualization tool for your data.
For every number $t$ the ecdf  $\hat F_n(t)$ is the fraction of data points
that are smaller than $t$.

\begin{gather*}
    \hat F_n(t) = \frac{ \mbox{number of elements in the sample} \leq t}n  
%\frac{1}{n} \sum_{i=1}^n \mathbf{1}\{x_i \le t\},
\end{gather*}
Obviously, just as the cdf the ecdf will always start 0 and end up at
1. You can easily read off quantiles from the ecdf, since the $y$--axis
represents the quantiles of the data. 

\subsubsection*{Example: ECDF of CCND3}
From the plot of the ecdf of CCND3, it can be seen that the
median of the overall data is roughly 1.8. The steep increase of the ecdf for
values greater than the median shows that larger values are more more common
in the data set.

\begin{knitrout}
\definecolor{shadecolor}{rgb}{0.969, 0.969, 0.969}\color{fgcolor}\begin{kframe}
\begin{alltt}
\hlkwd{plot}\hlstd{(}\hlkwd{ecdf}\hlstd{(golub[}\hlnum{1042}\hlstd{, ]) ,} \hlkwc{main} \hlstd{=} \hlstr{"ecdf of CCND3"}\hlstd{)}
\end{alltt}
\end{kframe}
\includegraphics[width=\maxwidth]{figure/ecdfCCND3} 

\end{knitrout}

In \Rpackage{ggplot2} there exists a specialized statistic which performs
the calculation necessary to obtain the ecdf. It also easy to split the
ecdf calculation by a factor, here we clearly see that for CCND3, 
the expression for ALL patients is always higher than the expression for
AML patients.

\begin{knitrout}
\definecolor{shadecolor}{rgb}{0.969, 0.969, 0.969}\color{fgcolor}\begin{kframe}
\begin{alltt}
\hlstd{ggECDF}\hlkwb{=}  \hlkwd{ggplot}\hlstd{(CCND3,} \hlkwd{aes}\hlstd{(}\hlkwc{x}\hlstd{=expCCND3,} \hlkwc{color} \hlstd{= patients))}
\hlstd{ggECDF} \hlopt{+} \hlkwd{stat_ecdf}\hlstd{(}\hlkwc{geom} \hlstd{=} \hlstr{"step"}\hlstd{,} \hlkwc{size} \hlstd{=} \hlnum{2}\hlstd{)} \hlopt{+} \hlkwd{ylab}\hlstd{(}\hlstr{"Frequency"}\hlstd{)}
\end{alltt}
\end{kframe}
\includegraphics[width=\maxwidth]{figure/ggecdfCCND3} 

\end{knitrout}


\subsection{Quantile--Quantile (QQ) Plot}
A method to compare the (e)cdf of  a data set to another (e)cdf is  
so--called quantile-quantile (Q--Q) plot. In such a plot, usually the quantiles of the
of the data set are displayed against the corresponding quantiles of
the normal distribution (bell--shaped). Hence, the empirical cdf of the
data are compared to the (corresponding) normal cdf.  \\

A straight line is added representing points which
correspond exactly to the quantiles of a corresponding normal distribution. 
By observing the extent in which the points appear on the line, it can be evaluated to
what degree the data are normally distributed. That is, the closer the  
values appear to the line, the more likely it is that the data are
normally distributed. To produce a Q--Q plot of the ALL gene expression values
of CCND3  one may use the following code \\


\begin{knitrout}
\definecolor{shadecolor}{rgb}{0.969, 0.969, 0.969}\color{fgcolor}\begin{kframe}
\begin{alltt}
\hlkwd{qqnorm}\hlstd{(golub[}\hlnum{1042}\hlstd{, gol.fac}\hlopt{==}\hlstr{"ALL"}\hlstd{])}
\hlkwd{qqline}\hlstd{(golub[}\hlnum{1042}\hlstd{, gol.fac}\hlopt{==}\hlstr{"ALL"}\hlstd{])}
\end{alltt}
\end{kframe}
\includegraphics[width=\maxwidth]{figure/QQCCND3} 

\end{knitrout}

From the resulting figure it can be observed that most of the data points
are on or near the straight line, while a few others are further away.
This is an expected behavior for gene expression data. The above example 
illustrates a case where the degree of non--normality
is moderate so that a clear conclusion cannot be drawn. \\

The QQ--plot can  also be used to compare the distributions of 
two data sets to one another by plotting their quantiles against each other. 
In order to do this, you can call the function \Rfunction{qqplot}. A comparison
of the distributions / ecdfs of AML and ALL gene expression values is given by:

\begin{knitrout}
\definecolor{shadecolor}{rgb}{0.969, 0.969, 0.969}\color{fgcolor}\begin{kframe}
\begin{alltt}
\hlkwd{qqplot}\hlstd{(golub[}\hlnum{1042}\hlstd{, gol.fac}\hlopt{==}\hlstr{"ALL"}\hlstd{], golub[}\hlnum{1042}\hlstd{, gol.fac}\hlopt{==}\hlstr{"AML"}\hlstd{])}
\end{alltt}
\end{kframe}
\includegraphics[width=\maxwidth]{figure/QQCCND3vs} 

\end{knitrout}

As we can see, the quantiles lie on a straight line, which indicates that
their general shape agrees. However, they might have different means and
standard deviations.


In \Rpackage{ggplot2} there exists a specialized statistic which performs
the calculation necessary to obtain the QQ--plot. The qnorm equivalent is
a bit harder to obtain, we have to extract the augmented data frame returned
by the \Rfunction{stat\textunderscore qq} function and add a linear smoother.

\begin{knitrout}
\definecolor{shadecolor}{rgb}{0.969, 0.969, 0.969}\color{fgcolor}\begin{kframe}
\begin{alltt}
\hlstd{ggQQ} \hlkwb{=}  \hlkwd{ggplot}\hlstd{(CCND3.ALL,} \hlkwd{aes}\hlstd{(}\hlkwc{sample}\hlstd{=CCND3.ALL))} \hlopt{+}  \hlkwd{stat_qq}\hlstd{()}
\hlcom{#ggQQ }
\hlstd{tp} \hlkwb{=} \hlkwd{cbind}\hlstd{(CCND3.ALL,} \hlkwd{as.data.frame}\hlstd{((}\hlkwd{print}\hlstd{(ggQQ)}\hlopt{$}\hlstd{data)[[}\hlnum{1}\hlstd{]][,} \hlkwd{c}\hlstd{(} \hlstr{"sample"}\hlstd{,} \hlstr{"theoretical"}\hlstd{)]))}
\end{alltt}
\end{kframe}
\includegraphics[width=\maxwidth]{figure/ggQQplotCCND31} 
\begin{kframe}\begin{alltt}
\hlstd{ggQQ} \hlopt{+} \hlkwd{stat_smooth}\hlstd{(}\hlkwc{mapping} \hlstd{=} \hlkwd{aes}\hlstd{(}\hlkwc{x} \hlstd{= theoretical,} \hlkwc{y} \hlstd{= sample),} \hlkwc{data} \hlstd{= tp,} \hlkwc{method} \hlstd{=} \hlstr{"lm"}\hlstd{,} \hlkwc{se} \hlstd{= F )}
\end{alltt}
\end{kframe}
\includegraphics[width=\maxwidth]{figure/ggQQplotCCND32} 

\end{knitrout}



\section{Descriptive Statistics}
There exist various ways to describe the central tendency as well as the spread
of data. In particular, the central tendency can be described by the mean or
the median, and the spread by the variance, standard deviation, interquartile
range, or median absolute deviation. These will be defined and illustrated.


\subsection{Measures of Central Tendency}
The most important descriptive statistics for central tendency are the mean
and the median. The sample mean of the data values $x_1, \dotsc, x_n$ is defined
as:

\begin{gather*}
\overline x =\frac{1}{k} \sum_{i=1}^k x_{i}
= \frac{1}{n} \left( x_{1} + \dotso + x_{n}\right).
\end{gather*}

Thus the sample mean is simply the average of the $n$ data values. Since it
is the sum of all data values divided by the sample size, a few extreme data
values may largely influence its size. In other words, the mean is not robust
against outliers. \\

The median is defined as the second quartile or the 50th percentile, and
is denoted by $x_{0.50}$. When the data are symmetrically distributed around the
mean, then the mean and the median are equal. Since extreme data values
do not influence the size of the median, it is very robust against outliers.
Robustness is important in biological applications because data are frequently contaminated 
by extreme or otherwise influential data values.

\subsubsection*{Example: Mean and Median of CCND3} 
To compute the mean and median of the ALL expression
values of gene CCND3 Cyclin D3 consider the following.


\begin{knitrout}
\definecolor{shadecolor}{rgb}{0.969, 0.969, 0.969}\color{fgcolor}\begin{kframe}
\begin{alltt}
\hlkwd{mean}\hlstd{(golub[}\hlnum{1042}\hlstd{, gol.fac}\hlopt{==}\hlstr{"ALL"}\hlstd{])}
\end{alltt}
\begin{verbatim}
#>   [1] 1.89
\end{verbatim}
\begin{alltt}
\hlkwd{median}\hlstd{(golub[}\hlnum{1042}\hlstd{, gol.fac}\hlopt{==}\hlstr{"ALL"}\hlstd{])}
\end{alltt}
\begin{verbatim}
#>   [1] 1.93
\end{verbatim}
\end{kframe}
\end{knitrout}

Note that the mean and the median do not differ much so that the distribution 
seems to be quite symmetric.

\subsection{Measures of Spread}
The most important measures of spread are the standard deviation, the interquartile range, 
and the median absolute deviation. The standard deviation
is the square root of the sample variance, which is defined as


\begin{gather*}
s^2 =\frac{1}{k - 1} \sum_{i=1}^k (x_{i} - \overline x)^{2}
= \frac{1}{n} \left( (x_{1} - \overline x)^{2} + \dotso + (x_{n} - \overline x)^{2}\right).
\end{gather*}

Hence, it is the average of the squared differences between the data values
and the sample mean. The sample standard deviation $s$ is the square root
of the sample variance and may be interpreted as the distance of the data
values to the mean. The variance and the standard deviation are not robust
against outliers. \\

The interquartile range is defined as the difference between the third and
the first quartile, that is $x_{0.75} - x_{0.25}$. It can be computed by the function
\Rfunction{IQR(x)}. More specifically, the value \Rfunction{IQR(x)/1.349}
 is a robust estimator of
the standard deviation.  \\

The median absolute deviation (MAD) is defined as
a constant times the median of the absolute deviations of the data from the
median. In \Rfunction{R} it is computed by the
function \Rfunction{mad} defined as the median of the sequence
$|x_1  - x_{0.5} |, \dotsc, |x_n - x_{0.5} |$
multiplied by the constant 1.4826. It equals the standard deviation in case
the data come from a bell--shaped (normal) distribution.
Because the interquartile range and the median absolute deviation are based
on quantiles, these are robust against outliers.


\subsubsection*{Example: Measures of Spread for CCND3} These measures of spread 
for the ALL expression values of gene CCND3 can be computed as follows.

\begin{knitrout}
\definecolor{shadecolor}{rgb}{0.969, 0.969, 0.969}\color{fgcolor}\begin{kframe}
\begin{alltt}
\hlkwd{sd}\hlstd{(golub[}\hlnum{1042}\hlstd{, gol.fac}\hlopt{==}\hlstr{"ALL"}\hlstd{])}
\end{alltt}
\begin{verbatim}
#>   [1] 0.491
\end{verbatim}
\begin{alltt}
\hlkwd{IQR}\hlstd{(golub[}\hlnum{1042}\hlstd{, gol.fac}\hlopt{==}\hlstr{"ALL"}\hlstd{])} \hlopt{/} \hlnum{1.349}
\end{alltt}
\begin{verbatim}
#>   [1] 0.284
\end{verbatim}
\begin{alltt}
\hlkwd{mad}\hlstd{(golub[}\hlnum{1042}\hlstd{, gol.fac}\hlopt{==}\hlstr{"ALL"}\hlstd{])}
\end{alltt}
\begin{verbatim}
#>   [1] 0.368
\end{verbatim}
\end{kframe}
\end{knitrout}

Due to the three outliers (cf. boxplot above) the standard deviation is larger
than the interquartile range and the mean absolute deviation. That is, the
absolute differences with respect to the median are somewhat smaller than
the root of the squared differences.

\subsubsection*{Exercise: Illustration of Mean and Standard Deviation}


\begin{enumerate}[label=(\emph{\alph*})]
 \item Compute the mean and the standard deviation for 1, 1.5, 2, 2.5, 3.
\item Compute the mean and the standard deviation for 1, 1.5, 2, 2.5, 30.
 \item Comment on the differences.
\end{enumerate}

\subsubsection*{Exercise: Plotting Gene Expressions of CCND3}
Use the gene expressions from "CCND3" of Golub collected in row
1042 of the object golub.
\begin{enumerate}[label=(\emph{\alph*})]

\item  Produce a  stripchart for the gene expressions separately
for the ALL as well as for the AML patients. Hint: Use a factor
for appropriate separation.
\item  Rotate the plot to a vertical position and keep it that way for the
questions to come.
\item  Color the ALL expressions red and AML blue. Hint: Use the \Robject{col}
parameter.
\item  Add a title to the plot. Hint: Use \Robject{title}.
\item   Change the boxes into stars. Hint: Use the \Robject{pch} parameter.
\end{enumerate}

\subsubsection*{Exercise: Plotting Gene Expressions using \CRANpkg{ggplot2}}
In this exercise we will plot the gene expressions for a couple
of genes in the golub data set in a single figure, using a separate 
panel for each gene.

\begin{enumerate}[label=(\emph{\alph*})]
\item Turn the golub data set into a 
data frame with genes in the columns. Add the
group descriptor \Robject{gol.fac} as a additional
column and turn it into a factor.

\item Select a random sample of 6 genes from the golub
data. HINT: Use the function \Rfunction{sample} to do this.

\item Melt the resulting data frame containing the selected genes
in such a way that all the gene expression values are in 
a single column.

\item Use this data frame and \CRANpkg{ggplot2} to produce
a pdf file containing boxplots separated per patient group for each
of the randomly selected genes. Also add the raw data points 
to the plot.
\end{enumerate}



\subsubsection*{Exercise: Comparing Normality for Two Genes} 
Consider the gene expression values in row 790 and 66 of the Golub data.

\begin{enumerate}[label=(\emph{\alph*})] 
 \item Produce a boxplot for the expression values of the ALL patients
and comment on the differences. Are there outliers?
\item Produce a QQ-plot and formulate a hypothesis about the normality of the genes.
 \item Compute the mean and the median for the expression values of
the ALL patients and compare these. Do this for both genes.
\end{enumerate}




%%%%%%%%%%%%%%%%%%%%%%%%%%%%%%%%%%%%%%%%%%%%%%%%%%%%%%%%%

\section{Answers to Exercises}

\subsubsection*{Exercise: Simple ggplot usage }
\begin{enumerate}[label=(\emph{\alph*})]

\item Use points as a geometry instead of lines
\item Use both lines and points
\item Add errorbars \Robject{geom\textunderscore errorbar} to the plot.
This requires  further aesthetics: 
\Robject{ymax} and \Robject{ymin}. 
The estimated error is stored in the variable \Robject{Sigma}.

\end{enumerate}

\subsubsection*{Solution: Simple ggplot usage }
\begin{knitrout}
\definecolor{shadecolor}{rgb}{0.969, 0.969, 0.969}\color{fgcolor}\begin{kframe}
\begin{alltt}
\hlcom{#a}
\hlcom{##########################################################}
\hlstd{plot_1} \hlkwb{<-} \hlkwd{ggplot}\hlstd{(}\hlkwd{aes}\hlstd{(} \hlkwc{x} \hlstd{= min,} \hlkwc{y} \hlstd{= Signal ),} \hlkwc{data} \hlstd{= proteins_pMek_sub)}
\hlstd{plot_1} \hlkwb{<-} \hlstd{plot_1} \hlopt{+} \hlkwd{geom_point}\hlstd{()}
\hlstd{plot_1} \hlkwb{<-} \hlstd{plot_1}  \hlopt{+} \hlkwd{xlab}\hlstd{(}\hlstr{"Time [min]"}\hlstd{)} \hlopt{+} \hlkwd{ylab}\hlstd{(}\hlstr{"pMEK Signal"}\hlstd{)}

\hlcom{#b}
\hlcom{##########################################################}
\hlstd{plot_1} \hlkwb{<-} \hlstd{plot_1} \hlopt{+} \hlkwd{geom_line}\hlstd{()}

\hlcom{#c}
\hlcom{##########################################################}
\hlstd{plot_1} \hlkwb{<-} \hlstd{plot_1} \hlopt{+} \hlkwd{geom_errorbar}\hlstd{(}\hlkwd{aes}\hlstd{(}\hlkwc{ymax} \hlstd{= Signal}\hlopt{+}\hlnum{2}\hlopt{*}\hlstd{Sigma,}
\hlkwc{ymin} \hlstd{= Signal}\hlopt{-}\hlnum{2}\hlopt{*}\hlstd{Sigma))}
\end{alltt}
\end{kframe}
\end{knitrout}


\subsubsection*{Exercise: ggplot faceting}
 
Using the data \Robject{proteins\textunderscore pMek\textunderscore sub}, 
to produce a plot split by the experimental condition factor using 
\Rfunction{facet\textunderscore wrap()}. 



\subsubsection*{Solution: ggplot faceting}
\begin{knitrout}
\definecolor{shadecolor}{rgb}{0.969, 0.969, 0.969}\color{fgcolor}\begin{kframe}
\begin{alltt}
\hlstd{(}\hlkwd{qplot}\hlstd{(min, Signal,} \hlkwc{data} \hlstd{= proteins_pMek,} \hlkwc{geom} \hlstd{=} \hlstr{"line"}\hlstd{,} \hlkwc{color} \hlstd{= Condition)}
\hlopt{+} \hlkwd{facet_wrap}\hlstd{(} \hlopt{~} \hlstd{Condition)} \hlopt{+} \hlkwd{geom_errorbar}\hlstd{(}\hlkwd{aes}\hlstd{(}\hlkwc{ymax} \hlstd{= Signal}\hlopt{+}\hlnum{2}\hlopt{*}\hlstd{Sigma,}
\hlkwc{ymin} \hlstd{= Signal}\hlopt{-}\hlnum{2}\hlopt{*}\hlstd{Sigma)))}
\end{alltt}
\end{kframe}
\end{knitrout}

\subsubsection*{Exercise: complex ggplot example}
 
Use the data \Robject{proteins}, to produce 
a plot of the time courses  split by the experimental 
target and colored according to the experimental 
conditions. Add error bars to your plot.



\subsubsection*{Solution: complex ggplot example}
\begin{knitrout}
\definecolor{shadecolor}{rgb}{0.969, 0.969, 0.969}\color{fgcolor}\begin{kframe}
\begin{alltt}
\hlstd{(}\hlkwd{qplot}\hlstd{(min, Signal,} \hlkwc{data} \hlstd{= proteins,} \hlkwc{geom} \hlstd{=} \hlstr{"line"}\hlstd{,} \hlkwc{color} \hlstd{= Condition)}
\hlopt{+} \hlkwd{facet_wrap}\hlstd{(} \hlopt{~} \hlstd{Target,} \hlkwc{ncol} \hlstd{=} \hlnum{2}\hlstd{)} \hlopt{+} \hlkwd{geom_errorbar}\hlstd{(}\hlkwd{aes}\hlstd{(}\hlkwc{ymax} \hlstd{= Signal}\hlopt{+}\hlnum{2}\hlopt{*}\hlstd{Sigma,}
                                                  \hlkwc{ymin} \hlstd{= Signal}\hlopt{-}\hlnum{2}\hlopt{*}\hlstd{Sigma)))}
\end{alltt}
\end{kframe}
\end{knitrout}


\subsubsection*{Exercise: Illustration of Mean and Standard Deviation}


\begin{enumerate}[label=(\emph{\alph*})] 
 \item Compute the mean and the standard deviation for 1, 1.5, 2, 2.5, 3.
\item Compute the mean and the standard deviation for 1, 1.5, 2, 2.5, 30.
 \item Comment on the differences.
\end{enumerate}

\subsubsection*{Solution:  Illustration of Mean and Standard Deviation}
\begin{knitrout}
\definecolor{shadecolor}{rgb}{0.969, 0.969, 0.969}\color{fgcolor}\begin{kframe}
\begin{alltt}
\hlcom{#Use x<- c(1,1.5,2,2.5,3) and mean(x) and sd(x) to obtain}
\hlcom{#that the mean is 2 and the standard deviation is 0.79}
(b) Now the mean is 7.4 and  the standard deviation dramatically increased 
(c) The outlier increased the mean as well as the standard deviation.
\end{alltt}
\end{kframe}
\end{knitrout}


\subsubsection*{Exercise: Plotting Gene Expressions of CCND3}
Use the gene expressions from "CCND3" of Golub collected in row
1042 of the object golub.
\begin{enumerate}[label=(\emph{\alph*})]

\item  Produce a so-called stripchart for the gene expressions separately
for the ALL as well as for the AML patients. Hint: Use a factor
for appropriate separation.
\item  Rotate the plot to a vertical position and keep it that way for the
questions to come.
\item  Color the ALL expressions red and AML blue. Hint: Use the \Robject{tcol}
parameter.
\item  Add a title to the plot. Hint: Use \Robject{title}.
\item   Change the boxes into stars. Hint: Use the \Robject{pch} parameter.
\end{enumerate}

\subsubsection*{Solution: Plotting Gene Expressions of CCND3}
\begin{knitrout}
\definecolor{shadecolor}{rgb}{0.969, 0.969, 0.969}\color{fgcolor}\begin{kframe}
\begin{alltt}
\hlstd{gol.fac} \hlkwb{<-} \hlkwd{factor}\hlstd{(golub.cl,}\hlkwc{levels}\hlstd{=}\hlnum{0}\hlopt{:}\hlnum{1}\hlstd{,} \hlkwc{labels}\hlstd{=} \hlkwd{c}\hlstd{(}\hlstr{"ALL"}\hlstd{,}\hlstr{"AML"}\hlstd{))}
\hlkwd{stripchart}\hlstd{(golub[}\hlnum{1042}\hlstd{,]} \hlopt{~} \hlstd{gol.fac,}\hlkwc{method}\hlstd{=}\hlstr{"jitter"}\hlstd{)}
\hlkwd{stripchart}\hlstd{(golub[}\hlnum{1042}\hlstd{,]} \hlopt{~} \hlstd{gol.fac,}\hlkwc{method}\hlstd{=}\hlstr{"jitter"}\hlstd{,}\hlkwc{vertical} \hlstd{=} \hlnum{TRUE}\hlstd{)}
\hlkwd{stripchart}\hlstd{(golub[}\hlnum{1042}\hlstd{,]} \hlopt{~} \hlstd{gol.fac,}\hlkwc{method}\hlstd{=}\hlstr{"jitter"}\hlstd{,}\hlkwc{col}\hlstd{=}\hlkwd{c}\hlstd{(}\hlstr{"red"}\hlstd{,} \hlstr{"blue"}\hlstd{),}
\hlkwc{vertical} \hlstd{=} \hlnum{TRUE}\hlstd{)}
\hlkwd{stripchart}\hlstd{(golub[}\hlnum{1042}\hlstd{,]} \hlopt{~} \hlstd{gol.fac,}\hlkwc{method}\hlstd{=}\hlstr{"jitter"}\hlstd{,}\hlkwc{col}\hlstd{=}\hlkwd{c}\hlstd{(}\hlstr{"red"}\hlstd{,} \hlstr{"blue"}\hlstd{),} \hlkwc{pch}\hlstd{=}\hlstr{"*"}
\hlstd{,}\hlkwc{vertical} \hlstd{=} \hlnum{TRUE}\hlstd{)}
\hlkwd{title}\hlstd{(}\hlstr{"CCND3 Cyclin D3 expression value for ALL and AMl patients"}\hlstd{)}
\end{alltt}
\end{kframe}
\end{knitrout}


\subsubsection*{Exercise: Plotting Gene Expressions using \CRANpkg{ggplot2}}
In this exercise we will plot the gene expressions for a couple
of genes in the golub data set in a single figure, using a separate 
panel for each gene.

\begin{enumerate}[label=(\emph{\alph*})]
\item Turn the golub data set into a 
data frame with genes in the columns. Add the
group descriptor \Robject{gol.fac} as a additional
column and turn it into a factor.

\item Select a random sample of 6 genes from the golub
data. HINT: Use the function \Rfunction{sample} to do this.

\item Melt the resulting data frame containing the selected genes
in such a way that all the gene expression values are in 
a single column.

\item Use this data frame and \CRANpkg{ggplot2} to produce
a pdf file containing boxplots separated per patient group for each
of the randomly selected genes. Also add the raw data points 
to the plot.
\end{enumerate}



\subsubsection*{Solution: Plotting Gene Expressions using \CRANpkg{ggplot2}}
\begin{knitrout}
\definecolor{shadecolor}{rgb}{0.969, 0.969, 0.969}\color{fgcolor}\begin{kframe}
\begin{alltt}
\hlcom{#a}
\hlcom{###############################################################}
\hlstd{golub.df} \hlkwb{=} \hlkwd{as.data.frame}\hlstd{(}\hlkwd{cbind}\hlstd{(}\hlkwd{t}\hlstd{(golub), gol.fac))}
\hlstd{golub.df}\hlopt{$}\hlstd{gol.fac} \hlkwb{=} \hlkwd{as.factor}\hlstd{(golub.df}\hlopt{$}\hlstd{gol.fac)}

\hlcom{#b}
\hlcom{###############################################################}
\hlstd{rand.sample} \hlkwb{<-} \hlkwd{c}\hlstd{(}\hlkwd{sample}\hlstd{(}\hlkwd{dim}\hlstd{(golub)[}\hlnum{1}\hlstd{],}\hlnum{6}\hlstd{),}\hlnum{3052}\hlstd{)}

\hlcom{#c}
\hlcom{###############################################################}
\hlstd{dataForPlot} \hlkwb{=} \hlkwd{melt}\hlstd{(golub.df[, rand.sample],}
    \hlkwc{id} \hlstd{=} \hlstr{'gol.fac'}\hlstd{)}

\hlcom{#e}
\hlcom{###############################################################}
\hlkwd{pdf}\hlstd{(}\hlstr{"boxplots.pdf"}\hlstd{,} \hlkwc{width} \hlstd{=} \hlnum{14}\hlstd{,} \hlkwc{height} \hlstd{=} \hlnum{10}\hlstd{)}
\hlstd{p} \hlkwb{<-} \hlkwd{qplot}\hlstd{(gol.fac, value,}  \hlkwc{data} \hlstd{= dataForPlot)} \hlopt{+}
  \hlkwd{geom_boxplot}\hlstd{(}\hlkwd{aes}\hlstd{(}\hlkwc{fill} \hlstd{= gol.fac))} \hlopt{+}
  \hlkwd{facet_wrap}\hlstd{(}\hlopt{~} \hlstd{variable)} \hlopt{+}
  \hlkwd{geom_point}\hlstd{(}\hlkwc{colour} \hlstd{=} \hlstr{'black'}\hlstd{,} \hlkwc{alpha} \hlstd{=} \hlnum{0.5}\hlstd{)}
\hlstd{p}
\hlkwd{dev.off}\hlstd{()}
\end{alltt}
\end{kframe}
\end{knitrout}

\subsubsection*{Exercise: Comparing Normality for Two Genes} 
Consider the gene expression values in row 790 and 66 of the Golub data.

\begin{enumerate}[label=(\emph{\alph*})] 
 \item Produce a boxplot for the expression values of the ALL patients
and comment on the differences. Are there outliers?
\item Produce a QQ-plot and formulate a hypothesis about the normality of the genes.
 \item Compute the mean and the median for the expression values of
the ALL patients and compare these. Do this for both genes.
\end{enumerate}

\subsubsection*{Solution: Comparing Normality for Two Genes}

\begin{knitrout}
\definecolor{shadecolor}{rgb}{0.969, 0.969, 0.969}\color{fgcolor}\begin{kframe}
\begin{alltt}
\hlcom{#Comparing two genes}
\hlcom{#(a) Use }
\hlkwd{boxplot}\hlstd{(golub[}\hlnum{66}\hlstd{,]}\hlopt{~}\hlstd{gol.fac)}
\hlkwd{dev.new}\hlstd{()}
\hlkwd{boxplot}\hlstd{(golub[}\hlnum{790}\hlstd{,]}\hlopt{~}\hlstd{gol.fac)}
\hlcom{#to observe that 790 has three}
\hlcom{#outliers and 66 has no outlier.}
\hlcom{# (dev.new() opens a new graphical window)}

\hlcom{#(b) Use }
\hlkwd{qqnorm}\hlstd{(golub[}\hlnum{66}\hlstd{,gol.fac}\hlopt{==}\hlstr{"ALL"}\hlstd{])}
\hlkwd{qqline}\hlstd{(golub[}\hlnum{66}\hlstd{,gol.fac}\hlopt{==}\hlstr{"ALL"}\hlstd{])}
\hlkwd{dev.new}\hlstd{()}
\hlkwd{qqnorm}\hlstd{(golub[}\hlnum{790}\hlstd{,gol.fac}\hlopt{==}\hlstr{"ALL"}\hlstd{])}
\hlkwd{qqline}\hlstd{(golub[}\hlnum{790}\hlstd{,gol.fac}\hlopt{==}\hlstr{"ALL"}\hlstd{])}
\hlcom{#to observe that nearly all values of 66 are on the line, where as for}
\hlcom{#790 the three outliers are way of the normality line. Hypothesis:}
\hlcom{#The expression values of 66 are normally distributed, but those of}
\hlcom{#row 790 are not.}

\hlcom{#(c) Use }
\hlkwd{mean}\hlstd{(golub[}\hlnum{66}\hlstd{,gol.fac}\hlopt{==}\hlstr{"ALL"}\hlstd{])}
\hlkwd{median}\hlstd{(golub[}\hlnum{790}\hlstd{,gol.fac}\hlopt{==}\hlstr{"ALL"}\hlstd{])}
\hlcom{#and# }
\hlkwd{mean}\hlstd{(golub[}\hlnum{790}\hlstd{,gol.fac}\hlopt{==}\hlstr{"ALL"}\hlstd{])}
\hlkwd{median}\hlstd{(golub[}\hlnum{790}\hlstd{,gol.fac}\hlopt{==}\hlstr{"ALL"}\hlstd{])}
\hlcom{#The mean (-1.174024) is larger than the median (-1.28137) due to}
\hlcom{#outliers on the right hand side. For the gene in row 66 the mean is}
\hlcom{#1.182503 and the median 1.23023. The differences are smaller.}
\end{alltt}
\end{kframe}
\end{knitrout}


\end{document}
