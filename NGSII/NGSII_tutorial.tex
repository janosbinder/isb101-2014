\documentclass{article}
\usepackage{minted}

\title{  NGS Tutorial}
\usepackage{amsmath}
\usepackage{minted}
\usepackage{natbib}
\usepackage{mathpazo}
%\usepackage{enumerate}
\usepackage{soul}
\usepackage{cases}
\setlength{\parindent}{0cm} 
\definecolor{bg}{rgb}{0.95,0.95,0.95}
\author{Roland Krause$^1$ \\[1em]Luxembourg Centre for Systems Biomedicine (LCSB),\\ Unversity of Luxembourg\\
\texttt{$^1$roland.krause@uni.lu}}


\usepackage{hyperref}
\begin{document}
\maketitle

\tableofcontents

\section{Introduction}

Exome sequencing is cost-effective way of sequencing 
by reducing the genome to the coding part by microarray hybridisation.

In this tutorial, we will map sequences using \verb+bwa+ and learn usual steps for quality improvements.
In order to speed up the tutorial, only chromosome 22 is taken into account.

First you will need to extract the sequences from a given .bam file.
\subsection{Set up}
Create directory \verb+ngs+ for all next-generation sequencing tutorials.
 
\begin{minted}{bash}
 	mkdir ngs
	cd ngs
\end{minted}





All data source data is kept in the directory \verb+/Users/roland.krause/Public/isb101/+. 

For your convenience, create a variable holding the path to the resources.

\begin{minted}{bash}
RESOURCE="/Users/roland.krause/Public/isb101/"
\end{minted}

Note: Not all commands are given in full in this tutorial. 
You might need to use the commands you have learned previously.

The programs \verb+samtools+ and \verb+bwa+ should be available from your path.

Question: How do you find out where a program is installed?
% which 

\section{Extracting reads from an existing BAM file}
\subsection{Copy the BAM file to your local folder}
This file has already been processed. We use it as source of our data.
The BAM file is called  \verb+daughter.Improved.bam+

Create a {\em soft link} to the file.

\begin{minted}{bash}
ln -s $RESOURCE/daughter.Improved.bam .
\end{minted}
% $
Questions:
\begin{enumerate}
\item Why don't we copy the file? 
\item Check the properties of the file using options of the \verb+ls+ command.
\item What happens if you would delete the link in your directory?
\item What happens if you delete the file in \verb|$RESOURCE| ?
\end{enumerate}
\subsection{Index the BAM file}
\begin{minted}{bash}
samtools index daughter.Improved.bam
\end{minted}
This will take a few seconds. 

Questions
\begin{enumerate}
\item  What did the command do? 
\item What is an {\em index}?
\end{enumerate}
\subsection{Visualize alignments with samtools tview }
\begin{minted}{bash}
samtools tview \
 daughter.Improved.bam \
 human_g1k_v37_Ensembl_MT_66.fasta
\end{minted}

\subsection{Extract chromosome 22 from the example BAM}
Slice chromosome 22 and save a piece in SAM format
\begin{minted}{bash}
samtools view daughter.Improved.bam 22 \
> daughter.Improved.22.sam
\end{minted}

\subsection{Convert SAM to FASTQ using PICARD}

Create a soft link to the picard-tools directory (!) in \verb+$RESCOURCE+ in your local \verb+ngs+ directory. 
\begin{minted}{bash}
ln -s $RESOURCE/picard-tools/ picard-tools
\end{minted}


Then, run the converter as follows:

\begin{minted}{bash}
java -jar picard-tools/SamToFastq.jar \
 I=daughter.Improved.22.sam \
 F=daughter.22.1.fq \
 F2=daughter.22.2.fq \
 VALIDATION_STRINGENCY=SILENT
\end{minted}

Inspect the output files and recapituliate the fastq-format.

\section{Performing quality control of the sequenced reads}
FastQC is a tool kit for quality performance. You will probably not be able to run 
this unless you are working from a Linux computer. 
If you are working from a Mac, you need to have X11 or XQuartz installed.

On the server login on a remote machine login via ssh with -X for X11 support.
\mint{bash}+ssh -X username@nitro.uni.lux+

the following command opens the FastQC GUI
\begin{minted}{bash}
perl FastQC/fastqc
\end{minted}

Load the new .sam file

We will discuss this together on screen. 

\section{Mapping}
\subsection{Indexing the reference }
The following command has to be use. This step is skipped as it takes to much time. The results can be found
in the \verb+$RESOURCE+ directory.
\begin{minted}{bash}
# bwa index -a bwtsw human_g1k_v37_Ensembl_MT_66.fasta
\end{minted}



\subsection{Perform alignment with Burrows-Wheeler Transform} 
In this main section of the mapping we will first align all reads and subsequently 
prepare the alignment for filtering and clean-up .

We will built indeces for further processing.
%%% -n 7 is used for a more sensitive alignment
%%% -q 15 is used for trimming sequences by quality (default=0=switched off)

Modify the path to the reference genome in the command line. 

Question: Why would a link not work in this case?

\begin{minted}{bash}
bwa mem -M 4 human_g1k_v37_Ensembl_MT_66.fasta \
daughter.22.1.fq daughter.22.2.fq \
> daughter.22.sam
\end{minted}
%  bwa mem -t 4 -M $RESOURCE/human_g1k_v37_Ensembl_MT_66.fasta daughter.22.1.fq daughter.22.2.fq > daughter.22.sam 

\subsection{Convert SAM to BAM}
\begin{minted}{bash}
samtools view -bS daughter.22.sam \
> daughter.22.bam
\end{minted}
\subsection{Sort BAM}
The suffix bam is automatically attached. This is for compatibility with PICARD and GATK.

\begin{minted}{bash}
samtools sort daughter.22.bam  daughter.22.sorted 
\end{minted}
% java7 -Xmx4g -XX:ParallelGCThreads=4 -Djava.io.tmpdir=./ -jar picard-tools//SortSam.jar INPUT= daughter.22.1.fq.bam OUTPUT= daughter.22.1.fq.sorted.bam SORT_ORDER=coordinate VALIDATION_STRINGENCY=LENIENT

%%%%%%%%%%%%%%%%%%%%%%%%%%%%%%%%%%%%%%%%%%%%
\subsection{Mark duplicate reads}
%%%%%%%%%%%%%%%%%%%%%%%%%%%%%%%%%%%%%%%%%%%%

Create a temporary folder and run picard tools.

Copy picard tools from the \verb+RESOURCE+ folder.
\begin{minted}{bash}
mkdir tmp

java -Djava.io.tmpdir=tmp -jar picard-tools/MarkDuplicates.jar \
 I=daughter.22.sorted.bam \
 O=daughter.22.sorted.marked.bam \
 METRICS_FILE=daughter.22.sorted.marked.metrics \
 VALIDATION_STRINGENCY=LENIENT

\end{minted}        
%%% useful links 
\url{http://picard.sourceforge.net/command-line-overview.shtml}
%MarkDuplicates


\url{http://sourceforge.net/apps/mediawiki/picard/index.php?title=Main_Page}
%Q:_How_does_MarkDuplicates_work.3F

\subsection{Add read-group}

This step is only necesssary for data generated around 2012.

\begin{minted}{bash}
java -jar picard-tools/AddOrReplaceReadGroups.jar \
INPUT=daughter.22.sorted.marked.bam \
OUTPUT=daughter.22.prepared.bam \
RGID=group1 RGLB= lib1 RGPL=illumina RGPU=unit1 RGSM=sample1

java -jar picard-tools/BuildBamIndex.jar INPUT=daughter.22.prepared.bam
\end{minted}

\section{Quality improvements}
Index the file with picard (Step A). 

% rm human_g1k_v37_Ensembl_MT_66.dict 
\begin{minted}{bash}
java -Xmx4g -jar picard-tools/CreateSequenceDictionary.jar \
R=human_g1k_v37_Ensembl_MT_66.fasta \
O=human_g1k_v37_Ensembl_MT_66.dict
\end{minted}           

Create an index on the reference sequence using samtools.


\begin{minted}{bash}
samtools faidx human_g1k_v37_Ensembl_MT_66.fasta
\end{minted}

\subsection{Realignment}
Create a index using GATK. 

%os.system("java -Xmx4g -jar %s/GenomeAnalysisTK.jar -T RealignerTargetCreator -R human_g1k_v37_Ense
%mbl_MT_66.fasta -o %s.bam.list -I %s.addrg_reads.bam --fix_misencoded_quality_scores "%(gatk, read_
%1, read_1))
Note that we use \verb+java7+ for running GATK, required for the latest version.

\begin{minted}{bash}
java7 -Xmx4g -jar GenomeAnalysisTK.jar \
-T RealignerTargetCreator \
-R human_g1k_v37_Ensembl_MT_66.fasta \
-o daughter.bam.list \
-I daughter.22.prepared.bam 
\end{minted}
% --fix_misencoded_quality_scores 
%java7 -Xmx4g -jar GenomeAnalysisTK.jar -T RealignerTargetCreator -R human_g1k_v37_Ensembl_MT_66.fasta -o daughter.22.1.fq.bam.list -I daughter.22.1.fq.addrg_reads.bam --fix_misencoded_quality_scores


\begin{minted}{bash} 
java7 -Xmx4g -Djava.io.tmpdir=./tmp/ -jar GenomeAnalysisTK.jar \
-T IndelRealigner \
-I daughter.22.prepared.bam \
-R human_g1k_v37_Ensembl_MT_66.fasta \
-targetIntervals daughter.bam.list -o daughter.22.real.bam
\end{minted} 
% --fix_misencoded_quality_scores 

%os.system("java -Xmx4g -Djava.io.tmpdir=./ -jar %s/GenomeAnalysisTK.jar -I %s.addrg_reads.bam -R human_g1k_v37_Ensembl_MT_66.fasta -T IndelRealigner -targetIntervals %s.bam.list -o %s.marked.realign ed.bam --fix_misencoded_quality_scores"%(gatk, read_1, read_1, read_1))

\subsection{BQSR(Base Quality Score Recalibration)}
\begin{minted}{bash}
 java7 -Xmx4g -jar GenomeAnalysisTK.jar \
   -T BaseRecalibrator \
   -I daughter.22.real.bam \
   -R human_g1k_v37_Ensembl_MT_66.fasta \
   -knownSites dbsnp_135.b37.vcf \
   -o recal_data.table
\end{minted}

\subsection{Fix the mate pairs }

%os.system("java -Djava.io.tmpdir=./flx-auswerter -jar %s/FixMateInformation.jar INPUT=%s.marked.realigned.bam OUTPUT=%s_bam.marked.realigned.fixed.bam SO=coordinate VALIDATION_STRINGENCY=LENIENT CREATE_INDEX=true "%(picard, read_1, read_1)


\begin{minted}{bash}
java7 -Djava.io.tmpdir=./tmp -jar picard-tools//FixMateInformation.jar INPUT=daughter.22.real.bam OUTPUT=daughter.22.real.fixed.bam SO=coordinate VALIDATION_STRINGENCY=LENIENT CREATE_INDEX=true 
\end{minted}
%%%%%%%%%%%%%%%%%%%%
\section{Variant calling}
%%%%%%%%%%%%%%%%%%%%

The final step of variant calling with the two tools most often used, \verb+samtools+ and \verb+GATK+.

\subsection{Samtools mpileup}
\begin{minted}{bash}
samtools mpileup \
 -S -E -g -Q 13 -q 20 \
 -f human_g1k_v37_Ensembl_MT_66.fasta \
 daughter.22.real.bam | \
 bcftools \
 view -vc - > daughter.22.mpileup.vcf
\end{minted}


Note: not working at the moment. 
\subsection{ GATK Unified Genotyper}
\begin{minted}{bash}
java7 -Djava.io.tmpdir=tmp -jar GenomeAnalysisTK.jar \
 -l INFO \
 -T UnifiedGenotyper \
 -R human_g1k_v37_Ensembl_MT_66.fasta \
 -I daughter.22.real.bam \
 -stand_call_conf 30.0 \
 -stand_emit_conf 10.0 \
 --genotype_likelihoods_model BOTH \
 --min_base_quality_score 13 \
 --max_alternate_alleles 3  \
 -A MappingQualityRankSumTest \
 -A AlleleBalance \
 -A BaseCounts \
 -A ChromosomeCounts \
 -A QualByDepth \
 -A ReadPosRankSumTest \
 -A MappingQualityZeroBySample \
 -A HaplotypeScore \
 -A LowMQ \
 -A RMSMappingQuality \
 -A BaseQualityRankSumTest \
 -L 22 \
 -o daughter.22.gatk.vcf
\end{minted} 

Questions:
\begin{enumerate}
\item How many variants are called?
\item Are both callers come up with the same variants? 
\item Inspect a case for indels and SNP and check those variants using
\verb+samtools tview+. Take screen shots.
\end{enumerate}

Next steps: Annotation and comparison of samples.

Acknowledgement: Holger Thiele, Kamel Jabbari (CCG Cologne), Patrick May, Dheeraj Bobbili (LCSB)
\end{document}
